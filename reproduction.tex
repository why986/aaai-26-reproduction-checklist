%File: anonymous-submission-latex-2026.tex
\documentclass[letterpaper]{article} % DO NOT CHANGE THIS
\usepackage[submission]{aaai2026}  % DO NOT CHANGE THIS
\usepackage{times}  % DO NOT CHANGE THIS
\usepackage{helvet}  % DO NOT CHANGE THIS
\usepackage{courier}  % DO NOT CHANGE THIS
\usepackage[hyphens]{url}  % DO NOT CHANGE THIS
\usepackage{graphicx} % DO NOT CHANGE THIS
\urlstyle{rm} % DO NOT CHANGE THIS
\def\UrlFont{\rm}  % DO NOT CHANGE THIS
\usepackage[authoryear]{natbib}  % DO NOT CHANGE THIS AND DO NOT ADD ANY OPTIONS TO IT
\usepackage{caption} % DO NOT CHANGE THIS AND DO NOT ADD ANY OPTIONS TO IT
\frenchspacing  % DO NOT CHANGE THIS
\setlength{\pdfpagewidth}{8.5in} % DO NOT CHANGE THIS
\setlength{\pdfpageheight}{11in} % DO NOT CHANGE THIS
%
% These are recommended to typeset algorithms but not required. See the subsubsection on algorithms. Remove them if you don't have algorithms in your paper.
\usepackage{algorithm}
\usepackage{algpseudocode}
\usepackage{amsmath}
\usepackage{amssymb}
\usepackage{amsthm}
\usepackage{amscd}
\usepackage{amsthm}

% These are are recommended to typeset listings but not required. See the subsubsection on listing. Remove this block if you don't have listings in your paper.
\usepackage{newfloat}
\usepackage{listings}
\DeclareCaptionStyle{ruled}{labelfont=normalfont,labelsep=colon,strut=off} % DO NOT CHANGE THIS
\lstset{%
	basicstyle={\footnotesize\ttfamily},% footnotesize acceptable for monospace
	numbers=left,numberstyle=\footnotesize,xleftmargin=2em,% show line numbers, remove this entire line if you don't want the numbers.
	aboveskip=0pt,belowskip=0pt,%
	showstringspaces=false,tabsize=2,breaklines=true}
\floatstyle{ruled}
\newfloat{listing}{tb}{lst}{}
\floatname{listing}{Listing}
%
% Keep the \pdfinfo as shown here. There's no need
% for you to add the /Title and /Author tags.
\pdfinfo{
/TemplateVersion (2026.1)
}

% DISALLOWED PACKAGES
% \usepackage{authblk} -- This package is specifically forbidden
% \usepackage{balance} -- This package is specifically forbidden
% \usepackage{color (if used in text)
% \usepackage{CJK} -- This package is specifically forbidden
% \usepackage{float} -- This package is specifically forbidden
% \usepackage{flushend} -- This package is specifically forbidden
% \usepackage{fontenc} -- This package is specifically forbidden
% \usepackage{fullpage} -- This package is specifically forbidden
% \usepackage{geometry} -- This package is specifically forbidden
% \usepackage{grffile} -- This package is specifically forbidden
% \usepackage{hyperref} -- This package is specifically forbidden
% \usepackage{navigator} -- This package is specifically forbidden
% (or any other package that embeds links such as navigator or hyperref)
% \indentfirst} -- This package is specifically forbidden
% \layout} -- This package is specifically forbidden
% \multicol} -- This package is specifically forbidden
% \nameref} -- This package is specifically forbidden
% \usepackage{savetrees} -- This package is specifically forbidden
% \usepackage{setspace} -- This package is specifically forbidden
% \usepackage{stfloats} -- This package is specifically forbidden
% \usepackage{tabu} -- This package is specifically forbidden
% \usepackage{titlesec} -- This package is specifically forbidden
% \usepackage{tocbibind} -- This package is specifically forbidden
% \usepackage{ulem} -- This package is specifically forbidden
% \usepackage{wrapfig} -- This package is specifically forbidden
% DISALLOWED COMMANDS
% \nocopyright -- Your paper will not be published if you use this command
% \addtolength -- This command may not be used
% \balance -- This command may not be used
% \baselinestretch -- Your paper will not be published if you use this command
% \clearpage -- No page breaks of any kind may be used for the final version of your paper
% \columnsep -- This command may not be used
% \newpage -- No page breaks of any kind may be used for the final version of your paper
% \pagebreak -- No page breaks of any kind may be used for the final version of your paperr
% \pagestyle -- This command may not be used
% \tiny -- This is not an acceptable font size.
% \vspace{- -- No negative value may be used in proximity of a caption, figure, table, section, subsection, subsubsection, or reference
% \vskip{- -- No negative value may be used to alter spacing above or below a caption, figure, table, section, subsection, subsubsection, or reference

\setcounter{secnumdepth}{2} %May be changed to 1 or 2 if section numbers are desired.

% The file aaai2026.sty is the style file for AAAI Press
% proceedings, working notes, and technical reports.
%

% Title

% Your title must be in mixed case, not sentence case.
% That means all verbs (including short verbs like be, is, using,and go),
% nouns, adverbs, adjectives should be capitalized, including both words in hyphenated terms, while
% articles, conjunctions, and prepositions are lower case unless they
% directly follow a colon or long dash
\author{
    %Authors
    % All authors must be in the same font size and format.
    Written by AAAI Press Staff\textsuperscript{\rm 1}\thanks{With help from the AAAI Publications Committee.}\\
    AAAI Style Contributions by Pater Patel Schneider,
    Sunil Issar,\\
    J. Scott Penberthy,
    George Ferguson,
    Hans Guesgen,
    Francisco Cruz\equalcontrib,
    Marc Pujol-Gonzalez\equalcontrib
}
\affiliations{
    %Afiliations
    \textsuperscript{\rm 1}Association for the Advancement of Artificial Intelligence\\
    % If you have multiple authors and multiple affiliations
    % use superscripts in text and roman font to identify them.
    % For example,

    % Sunil Issar\textsuperscript{\rm 2},
    % J. Scott Penberthy\textsuperscript{\rm 3},
    % George Ferguson\textsuperscript{\rm 4},
    % Hans Guesgen\textsuperscript{\rm 5}
    % Note that the comma should be placed after the superscript

    1101 Pennsylvania Ave, NW Suite 300\\
    Washington, DC 20004 USA\\
    % email address must be in roman text type, not monospace or sans serif
    proceedings-questions@aaai.org
%
% See more examples next
}

%Example, Single Author, ->> remove \iffalse,\fi and place them surrounding AAAI title to use it
\iffalse
\title{My Publication Title --- Single Author}
\author {
    Author Name
}
\affiliations{
    Affiliation\\
    Affiliation Line 2\\
    name@example.com
}
\fi

\iffalse
%Example, Multiple Authors, ->> remove \iffalse,\fi and place them surrounding AAAI title to use it
\title{My Publication Title --- Multiple Authors}
\author {
    % Authors
    First Author Name\textsuperscript{\rm 1},
    Second Author Name\textsuperscript{\rm 2},
    Third Author Name\textsuperscript{\rm 1}
}
\affiliations {
    % Affiliations
    \textsuperscript{\rm 1}Affiliation 1\\
    \textsuperscript{\rm 2}Affiliation 2\\
    firstAuthor@affiliation1.com, secondAuthor@affilation2.com, thirdAuthor@affiliation1.com
}
\fi


% REMOVE THIS: bibentry
% This is only needed to show inline citations in the guidelines document. You should not need it and can safely delete it.
\usepackage{bibentry}
\usepackage{color}
% END REMOVE bibentry

\begin{document}



\section*{Reproduction Checklist}
\subsection*{Methodology Description}
\begin{itemize}
    \item Includes a conceptual outline and/or pseudocode description of AI methods introduced. \\ \makebox[\linewidth][r]{  (\textit{yes / partial / no / NA})}
    \item Clearly delineates statements that are opinions, hypothesis, and speculation from objective facts and results. \\ \makebox[\linewidth][r]{ (\textit{yes / no})}
    \item Provides well marked pedagogical references for less-familiar readers to gain background necessary to replicate the paper. \leavevmode \hfill\makebox[0pt][r]{(\textit{yes / no})}
\end{itemize}

\subsection*{Theoretical Contributions}
\begin{itemize}
    \item Does this paper make theoretical contributions? \\
     \makebox[\linewidth][r]{ (\textit{yes / no})} \\
    \textit{If yes, please complete the list below.}
    \begin{itemize}
        \item All assumptions and restrictions are stated clearly and formally. \hfill (\textit{yes / partial / no})
        \item All novel claims are stated formally (e.g., in theorem statements). \hfill (\textit{yes / partial / no})
        \item Proofs of all novel claims are included. \\ \makebox[\linewidth][r]{ (\textit{yes / partial / no})}
        \item Proof sketches or intuitions are given for complex and/or novel results. \hfill (\textit{yes / partial / no})
        \item Appropriate citations to theoretical tools used are given. \hfill (\textit{yes / partial / no})
        \item All theoretical claims are demonstrated empirically to hold. \hfill (\textit{yes / partial / no / NA})
        \item All experimental code used to eliminate or disprove claims is included. \hfill (\textit{yes / no / NA})
    \end{itemize}
\end{itemize}

\subsection*{Datasets}
\begin{itemize}
    \item Does this paper rely on one or more datasets? \hfill (\textit{yes / no}) \\
    \textit{If yes, please complete the list below.}
    \begin{itemize}
        \item A motivation is given for why the experiments are conducted on the selected datasets. \\ \makebox[\linewidth][r]{ (\textit{yes / partial / no / NA})}
        \item All novel datasets introduced in this paper are included in a data appendix. \hfill (\textit{yes / partial / no / NA})
        \item All novel datasets introduced in this paper will be made publicly available upon publication of the paper with a license that allows free usage for research purposes. \\ \makebox[\linewidth][r]{ (\textit{yes / partial / no / NA})}
        \item All datasets drawn from the existing literature are accompanied by appropriate citations. \\ \makebox[\linewidth][r]{ (\textit{yes / no / NA})}
        \item All datasets drawn from the existing literature are publicly available. \\ \makebox[\linewidth][r]{ (\textit{yes / partial / no / NA})}
        \item All datasets that are not publicly available are described in detail, with explanation why publicly available alternatives are not scientifically satisficing. \\ \makebox[\linewidth][r]{ (\textit{yes / partial / no / NA})}
    \end{itemize}
\end{itemize}

\subsection*{Computational Experiments}
\begin{itemize}
    \item Does this paper include computational experiments? \\ \makebox[\linewidth][r]{ (\textit{yes / no})} \\
    \textit{If yes, please complete the list below.}
    \begin{itemize}
        \item This paper states the number and range of values tried per (hyper-)parameter during development, along with the criterion used for selecting the final parameter setting. \\ \makebox[\linewidth][r]{ (\textit{yes / partial / no / NA})}
        \item Any code required for pre-processing data is included in the appendix. \hfill (\textit{yes / partial / no})
        \item All source code required for conducting and analyzing the experiments is included in a code appendix. \\ \makebox[\linewidth][r]{ (\textit{yes / partial / no})}
        \item All source code required for conducting and analyzing the experiments will be made publicly available upon publication of the paper with a license that allows free usage for research purposes. \hfill (\textit{yes / partial / no})
        \item All source code implementing new methods have comments detailing the implementation, with references to the paper where each step comes from. \\ \makebox[\linewidth][r]{ (\textit{yes / partial / no})}
        \item If an algorithm depends on randomness, then the method used for setting seeds is described in a way sufficient to allow replication of results. \\ \makebox[\linewidth][r]{ (\textit{yes / partial / no / NA})}
        \item This paper specifies the computing infrastructure used for running experiments (hardware and software). \\ \makebox[\linewidth][r]{ (\textit{yes / partial / no})}
        \item This paper formally describes evaluation metrics used and explains the motivation for choosing these metrics. \\ \makebox[\linewidth][r]{ (\textit{yes / partial / no})}
        \item This paper states the number of algorithm runs used to compute each reported result. \hfill (\textit{yes / no})
        \item Analysis of experiments goes beyond single-dimensional summaries of performance to include measures of variation, confidence, or other distributional information. \hfill\makebox[0pt][r]{ (\textit{yes / no})}
        \item The significance of any improvement or decrease in performance is judged using appropriate statistical tests (e.g., Wilcoxon signed-rank). \hfill (\textit{yes / partial / no})
        \item This paper lists all final (hyper-)parameters used for each model/algorithm in the paper’s experiments. \\ \makebox[\linewidth][r]{ (\textit{yes / partial / no / NA})}
    \end{itemize}
\end{itemize}




\end{document}